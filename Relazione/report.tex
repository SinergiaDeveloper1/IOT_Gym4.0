%%%%%%%%%%%%%%%%%%%%%%%%%%%%%%%%%%%%%%%%%
% Stylish Article
% LaTeX Template
% Version 2.2 (2020-10-22)
%
% This template has been downloaded from:
% http://www.LaTeXTemplates.com
%
% Original author:
% Mathias Legrand (legrand.mathias@gmail.com) 
% With extensive modifications by:
% Vel (vel@latextemplates.com)
%
% License:
% CC BY-NC-SA 3.0 (http://creativecommons.org/licenses/by-nc-sa/3.0/)
%
%%%%%%%%%%%%%%%%%%%%%%%%%%%%%%%%%%%%%%%%%

%----------------------------------------------------------------------------------------
%	PACKAGES AND OTHER DOCUMENT CONFIGURATIONS
%----------------------------------------------------------------------------------------

\documentclass[fleqn,10pt]{SelfArx} % Document font size and equations flushed left

%\usepackage[english]{babel} % Specify a different language here - english by default
\usepackage[italian]{babel} % Specify a different language here - english by default

\usepackage{lipsum} % Required to insert dummy text. To be removed otherwise

%----------------------------------------------------------------------------------------
%	COLUMNS
%----------------------------------------------------------------------------------------

\setlength{\columnsep}{0.55cm} % Distance between the two columns of text
\setlength{\fboxrule}{0.75pt} % Width of the border around the abstract

%----------------------------------------------------------------------------------------
%	COLORS
%----------------------------------------------------------------------------------------

\definecolor{color1}{RGB}{0,0,90} % Color of the article title and sections
\definecolor{color2}{RGB}{0,20,20} % Color of the boxes behind the abstract and headings

%----------------------------------------------------------------------------------------
%	HYPERLINKS
%----------------------------------------------------------------------------------------

\usepackage{hyperref} % Required for hyperlinks

\hypersetup{
	hidelinks,
	colorlinks,
	breaklinks=true,
	urlcolor=color2,
	citecolor=color1,
	linkcolor=color1,
	bookmarksopen=false,
	pdftitle={Title},
	pdfauthor={Author},
}

%----------------------------------------------------------------------------------------
%	ARTICLE INFORMATION
%----------------------------------------------------------------------------------------

\JournalInfo{Programmazione per l'IoT - Laurea Magistrale in Informatica Applicata - DiSPeA - Università degli Studi di Urbino Carlo Bo} % Journal information
\Archive{Data di pubblicazione xx/xx/xxxx - DOI: xxxx/xxxxxxx} % Informazioni che verranno inserite dal Docente in fase di pubblicazione

\PaperTitle{4.0 Gym Prototype} % Article title

\Authors{Luca Cinti\textsuperscript{1}*, Emanuele Lattanzi\textsuperscript{2}} % Il docente Emanuele Lattanzi figura come autore
% al fine di poter gestire la procedura di submission sul repository pubblico (nell'affiliazione viene chiarito il ruolo)
\affiliation{\textsuperscript{1}\textit{Laurea Magistrale in Informatica Applicata, Università degli Studi di Urbino Carlo Bo, Urbino, Italia}} % Author affiliation
\affiliation{\textsuperscript{2}\textit{Docente di Programmazione per l'Internet of Things, Università degli Studi di Urbino Carlo Bo, Urbino, Italia}} % Author affiliation
\affiliation{*\textbf{Corresponding author}: l.cinti@campus.uniurb.it} % Corresponding author

\Keywords{IOT --- Arduino --- ESP32 --- MQTT --- Gym Metrics} % Keywords - if you don't want any simply remove all the text between the curly brackets
\newcommand{\keywordname}{Keywords} % Defines the keywords heading name

%----------------------------------------------------------------------------------------
%	ABSTRACT
%----------------------------------------------------------------------------------------

\Abstract{
	In questo elaborato abbiamo cercato di predisporre un'architettura per la raccolta dei dati
	in una piccola home gym, innanzitutto per verificare la fattibilità del reperimento di alcune misurazioni, 
	per poi passare all'analisi dei dati raccolti, sia per estrapolare delle metriche significative ai fini dell'allenamento, 
	che delle caratteristiche termiche dell'ambiente.
}

%----------------------------------------------------------------------------------------

\begin{document}

\maketitle % Output the title and abstract box

%\tableofcontents % Output the contents section

\thispagestyle{empty} % Removes page numbering from the first page

%----------------------------------------------------------------------------------------
%	ARTICLE CONTENTS
%----------------------------------------------------------------------------------------

\section*{Introduzione} % The \section*{} command stops section numbering
%\lipsum[1-3] % Dummy text
% and some mathematics $\cos\pi=-1$ and $\alpha$ in the text\footnote{And some mathematics $\cos\pi=-1$ and $\alpha$ in the text.}
%Questa è una citazione al libro di testo~\cite{milenkovic2020internet}.

L'obiettivo preliminare di questo elaborato, era verificare la fattibilità dell'allestimento di un sistema di raccolta dati in una piccola home gym, 
costruendo l'infrastruttura per 

Come primo obiettivo di questo elaborato, ci siamo chiesti se fosse possibile raccogliere dei dati più o meno interessanti in termini di temperatura, umidità e 

%------------------------------------------------

\section{Methods}

\begin{figure*}[ht]\centering % Using \begin{figure*} makes the figure take up the entire width of the page
	\includegraphics[width=\linewidth]{view}
	\caption{Wide Picture}
	\label{fig:view}
\end{figure*}

\lipsum[4] % Dummy text

\begin{equation}
	\cos^3 \theta =\frac{1}{4}\cos\theta+\frac{3}{4}\cos 3\theta
	\label{eq:refname2}
\end{equation}

\lipsum[5] % Dummy text

\begin{enumerate}[noitemsep] % [noitemsep] removes whitespace between the items for a compact look
	\item First item in a list
	\item Second item in a list
	\item Third item in a list
\end{enumerate}

\subsection{Subsection}

\lipsum[6] % Dummy text

\paragraph{Paragraph} \lipsum[7] % Dummy text
\paragraph{Paragraph} \lipsum[8] % Dummy text

\subsection{Subsection}

\lipsum[9] % Dummy text

\begin{figure}[ht]\centering
	\includegraphics[width=\linewidth]{results}
	\caption{In-text Picture}
	\label{fig:results}
\end{figure}

Reference to Figure \ref{fig:results}.

%------------------------------------------------

\section{Risultati}

\lipsum[10] % Dummy text

\subsection{Subsection}

\lipsum[11] % Dummy text

\begin{table}[hbt]
	\caption{Table of Grades}
	\centering
	\begin{tabular}{llr}
		\toprule
		\multicolumn{2}{c}{Name} \\
		\cmidrule(r){1-2}
		First name & Last Name & Grade \\
		\midrule
		John & Doe & $7.5$ \\
		Richard & Miles & $2$ \\
		\bottomrule
	\end{tabular}
	\label{tab:label}
\end{table}

\subsubsection{Subsubsection}

\lipsum[12] % Dummy text

\begin{description}
	\item[Word] Definition
	\item[Concept] Explanation
	\item[Idea] Text
\end{description}

\subsubsection{Subsubsection}

\lipsum[13] % Dummy text

\begin{itemize}[noitemsep] % [noitemsep] removes whitespace between the items for a compact look
	\item First item in a list
	\item Second item in a list
	\item Third item in a list
\end{itemize}

\subsubsection{Subsubsection}

\lipsum[14] % Dummy text

\section{Conclusioni}

\lipsum[15-23] % Dummy text

%----------------------------------------------------------------------------------------
%	REFERENCE LIST
%----------------------------------------------------------------------------------------

\phantomsection
\bibliographystyle{unsrt}
\bibliography{sample.bib}

%----------------------------------------------------------------------------------------

\end{document}
